\documentclass{article}
\usepackage[utf8]{inputenc}
\usepackage{pythonhighlight}
\usepackage{listings}
\usepackage{minted}

\title{Argparse}
\author{MJ Lukiman}
\date{July 2019}

\begin{document}

\maketitle

\section{Motivation}

Instead of using \mintinline{python}{sys.argv}, which is admittedly very simple, I've seen many programs using \mintinline{python}{argparse}. I think it would behoove me to learn it.

\section{Advantages over sys.argv}
Argparse automatically generates user help errors, and can constrain the type of input the program ingests.

\section{The steps in every file}

While sys.argv can be written in one line, argparse takes a couple more lines of setup. But, I assume that this ends up saving time in the long run.\\
First, we create the parser object with the description of the program.
\begin{python}
import argparse as ap
p = ap.ArgumentParser(description='Make some meaningful arguments')
\end{python}
Then we add arguments in anticipation of calling \mintinline{python}{p.parse_args()}. We add arguments like so:
\begin{python}
parse.add_argument('TEXT_OF_ARGUMENT', dest='CUSTOM_ATTRIBUTE_NAME', const='ACTIVE_VALUE', default='OTHERWISE_VALUE', type=TYPE_CONVERSION, choices=TYPE_CONSTRAINTS, nargs=AUTOMATON, required=REQUIRED_BOOL, help='HELP_DESCRIPTION')
\end{python}
Now, we can call \mintinline{python}{ap.parse_args()} to automatically process the inputs given to the script. For testing purposes we can simulate inputs via a list of argument strings a la:
\begin{python}
ap.parse_args(['arg1','arg2',...]
\end{python}
Presence and absence or inverse of arguments can be indicated using \begin{python}
action='store_true.
\end{python}
Several short options can be joined together, using only a single - prefix, as long as only the last option (or none of them) requires a value:

\begin{python}
>>> parser = argparse.ArgumentParser(prog='PROG')
>>> parser.add_argument('-x', action='store_true')
>>> parser.add_argument('-y', action='store_true')
>>> parser.add_argument('-z')
>>> parser.parse_args(['-xyzZ'])
Namespace(x=True, y=True, z='Z')
\end{python}
\section{Accessing the namespace}
\mintinline{python}{ap.parse_args()} would return an object whose properties can be accessed. It can also be turned into a dictionary like:
\begin{python}
args = ap.parse_args()
\end{python}
To add to an already existing object, just define the name space in:
\begin{python}
ap.parse_args(namespace=OBJECT)
\end{python}

\end{document}
